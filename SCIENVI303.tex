% Options for packages loaded elsewhere
\PassOptionsToPackage{unicode}{hyperref}
\PassOptionsToPackage{hyphens}{url}
%
\documentclass[
]{article}
\usepackage{amsmath,amssymb}
\usepackage{lmodern}
\usepackage{iftex}
\ifPDFTeX
  \usepackage[T1]{fontenc}
  \usepackage[utf8]{inputenc}
  \usepackage{textcomp} % provide euro and other symbols
\else % if luatex or xetex
  \usepackage{unicode-math}
  \defaultfontfeatures{Scale=MatchLowercase}
  \defaultfontfeatures[\rmfamily]{Ligatures=TeX,Scale=1}
\fi
% Use upquote if available, for straight quotes in verbatim environments
\IfFileExists{upquote.sty}{\usepackage{upquote}}{}
\IfFileExists{microtype.sty}{% use microtype if available
  \usepackage[]{microtype}
  \UseMicrotypeSet[protrusion]{basicmath} % disable protrusion for tt fonts
}{}
\makeatletter
\@ifundefined{KOMAClassName}{% if non-KOMA class
  \IfFileExists{parskip.sty}{%
    \usepackage{parskip}
  }{% else
    \setlength{\parindent}{0pt}
    \setlength{\parskip}{6pt plus 2pt minus 1pt}}
}{% if KOMA class
  \KOMAoptions{parskip=half}}
\makeatother
\usepackage{xcolor}
\usepackage[margin=1in]{geometry}
\usepackage{graphicx}
\makeatletter
\def\maxwidth{\ifdim\Gin@nat@width>\linewidth\linewidth\else\Gin@nat@width\fi}
\def\maxheight{\ifdim\Gin@nat@height>\textheight\textheight\else\Gin@nat@height\fi}
\makeatother
% Scale images if necessary, so that they will not overflow the page
% margins by default, and it is still possible to overwrite the defaults
% using explicit options in \includegraphics[width, height, ...]{}
\setkeys{Gin}{width=\maxwidth,height=\maxheight,keepaspectratio}
% Set default figure placement to htbp
\makeatletter
\def\fps@figure{htbp}
\makeatother
\setlength{\emergencystretch}{3em} % prevent overfull lines
\providecommand{\tightlist}{%
  \setlength{\itemsep}{0pt}\setlength{\parskip}{0pt}}
\setcounter{secnumdepth}{-\maxdimen} % remove section numbering
\ifLuaTeX
  \usepackage{selnolig}  % disable illegal ligatures
\fi
\IfFileExists{bookmark.sty}{\usepackage{bookmark}}{\usepackage{hyperref}}
\IfFileExists{xurl.sty}{\usepackage{xurl}}{} % add URL line breaks if available
\urlstyle{same} % disable monospaced font for URLs
\hypersetup{
  pdftitle={SCIENVI301 Ecology Spring 2023},
  hidelinks,
  pdfcreator={LaTeX via pandoc}}

\title{SCIENVI301EcologySpring 2023}
\author{}
\date{\vspace{-2.5em}Last updated: 2023-01-26}

\begin{document}
\maketitle

{
\setcounter{tocdepth}{3}
\tableofcontents
}
\hfill\break

\hypertarget{introduction}{%
\section{Introduction}\label{introduction}}

This page collects workshop materials, data sources and supplementary
materials for the data-driven assignments in the Spring 2023 edition of
Ecology at UCR. You can also access the related files directly on
\href{https://github.com/ucrdatacenter/projects/tree/main/SCIENVI201/2022h1}{Github}.

If you have any questions related to the assignment do not hesitate to
reach Bianka during Data Center office hours (held every Wednesday 17:00
- 19:00 on
\href{https://universitycollegeroosevelt.zoom.us/j/2831128718?pwd=UmRuSzVqSTZyMndDbDRGSkV5VWFVQT09}{Zoom}).
There is no need to sign up in advance to join office hours. If this
time is inconvenient for you feel free to write Bianka an email
(\href{mailto:ucrdatacenter@ucr.nl}{\nolinkurl{ucrdatacenter@ucr.nl}})
to schedule an individual meeting.

\hfill\break

\hypertarget{working-with-r-and-rstudio}{%
\section{Working with R and RStudio}\label{working-with-r-and-rstudio}}

\hypertarget{recommended-reading}{%
\subsection{Recommended reading}\label{recommended-reading}}

It is highly recommended that you read the sections 2, 3.1-3.4 and 4-6
of
\href{https://github.com/ClaudiaBrauer/A-very-short-introduction-to-R/blob/master/documents/A\%20(very)\%20short\%20introduction\%20to\%20R.pdf}{A
(very) short introduction to R} and sections 1-4 of
\href{http://r-statistics.co/ggplot2-Tutorial-With-R.html\#6.1\%20Make\%20a\%20time\%20series\%20plot\%20(using\%20ggfortify)}{How
to make any plot in ggplot2?} before starting working with R.

\hypertarget{installing-r-and-rstudio}{%
\subsection{Installing R and RStudio}\label{installing-r-and-rstudio}}

R is the programming language that you use when working with data.
RStudio is a development environment that makes working with R much more
convenient. Both are free to download and install.

To download R, go to
\href{https://cloud.r-project.org/}{cloud.r-project.org} and follow the
instructions on the page. To download RStudio, go to
\href{https://www.rstudio.com/products/rstudio/download/}{rstudio.com/products/rstudio/download},
scroll down and download the file recommended for your operating system.

When installing, you can always stick to the default settings, unless
you have different preferences.

In case you get stuck at any point, or would like more guidance in the
installation process, feel free to check out any of the following links:

\begin{itemize}
\tightlist
\item
  \href{https://www.dataquest.io/blog/tutorial-getting-started-with-r-and-rstudio/}{Tutorial:
  Getting Started with R and RStudio}
\item
  \href{https://www.youtube.com/watch?v=0Qu7Jg1Jw5A}{R Tutorial: How to
  install R and R studio (video)}
\end{itemize}

\hypertarget{creating-a-project-in-rstudio}{%
\subsection{Creating a project in
RStudio}\label{creating-a-project-in-rstudio}}

It is convenient to create an R project for an assignment that you are
working on. A project is basically a folder that stores all files
related to the assignment.

You can create a project as follows:

\begin{itemize}
\tightlist
\item
  Open RStudio and click on ``Project: (None)'' in the top right corner.
\item
  Open the dropdown window and click on ``New Project\ldots.''
\item
  In the popup window select ``New Directory'', then ``New Project.''
\item
  Choose a sensible name for your project and enter it as the Directory
  Name. You can either use the default file path or change it by
  clicking ``Browse\ldots{}'' next to ``Create project as a subdirectory
  of:.''
\item
  Finally, click on ``Create project.''
\end{itemize}

After a project is created, there are two easy ways of accessing it. You
can either use the same dropdown window in the top right corner of
RStudio that you used to create the project, and click on the name of
the project there, or you can find the project folder within your files
and click on the file with the .Rproj extension.

\hfill\break

\hypertarget{simple-assignment}{%
\section{Simple Assignment}\label{simple-assignment}}

\hypertarget{research-assignment}{%
\section{Research Assignment}\label{research-assignment}}

\end{document}
